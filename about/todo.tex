\documentclass{article}

\usepackage[top=1in,left=1in,right=1in,bottom=1in]{geometry}

\title{\textbf{Research Question Peer Review Tool}}
\author{George Mason Peer Research Mentor Production Team}
\date{\today}

\begin{document}

\maketitle

\section{Background}

Outside of some required meetings with the GTA or professor, a lot of HNRS
110 students don't get very much feedback on their research questions.
However, narrowing down your research topic to a question can be extremely
difficult and one of the most important processes to guarantee a successful
paper. Since professors, GTA, and PRMs don't have unlimited time, and the
fact that more sets of eyes can generally give the best review of a tough
question, it's clear that peer review is the best method of allowing
students to get in-depth feedback and commentary on their topics/questions
as they move forward in the research process.

\section{Project Overview}
This project is essentially a crowd-sourcing tool to allow HNRS 110
students to provide peer review and gain feedback on their Research Topics
and Questions as they are narrowing their focus and moving into the writing
process.

\subsection{Research Topics \& Questions}
Students will be able to submit their research topics or questions to a
community database of student submissions. This database will be accessible
directly by PRMs and HNRS 110 staff.

\subsection{Student Feedback}
\paragraph{Comments} Students will be able to submit comments on the topics
in the database, providing detailed feedback specifically tailored to a
single student's research topic or question.

\paragraph{Rating} Students will also be able to rate topics/questions up
and down. This will be slightly obscured from the user in that you don't
just up-/down- vote an entry, but rate it as "Good" or "Needs Work"
corresponding to +1/-1, respectively. The rating system will determine the
order in which questions appear.

\subsection{Moderator Interface}
PRMs will have access to a special, protected ``moderator'' interface
wherein they can monitor and control certain aspects of the student
discussion. They can remove objectionable comments or research
topics/questions from the database if need be. Additionally, they can alter
question ratings to move them up or down. Moderators can ``sticky''
particularly interesting or good questions to the top of the listing.

\subsection{Instructions}
This project will also offer to students a comprehensive textual guide
describing what is expected of them. This includes topics such as ``what
makes a good research question'' and ``what is constructive criticism.''
Hopefully this documentation will serve as a solid reference for students
when they just want suggestions on what their research topic/question
should look like, as well as what is expected of them when submitting
feedback on other students' work.

\section{ToDo}

\begin{verbatim}
  - fill out instructions page
    + in-depth description of what a good research question is
    + list of traits of good/bad questions
    + description of good/bad feedback
      o emphasize accountability + what is expected as constructive
      criticism
  - continue development of feedback page
    + better comment display
    + rating of comments?
    + include username alongside comments to provide accountability
  - continue development of moderator interface
    + admin registration
    + add/remove comments
    + add/remove questions
    + alter question ratings
    + "sticky" good questions
\end{verbatim}

\end{document}
